\documentclass[a4paper]{article}

\usepackage[utf8]{inputenc}
\usepackage[T1]{fontenc}
\usepackage[french]{babel}
\usepackage{hyperref}

\hypersetup{
    colorlinks,
    citecolor=black,
    filecolor=black,
    linkcolor=black,
    urlcolor=black
}

\usepackage[a4paper,left=2cm,right=2cm]{geometry}
\usepackage{graphicx}

\title{Devoir Maison de\\
  Modélisation et Programmation Objet}          
\author{Cazorla Axel,\\
  Laviron Pablo\\  
  \href{https://github.com/Ethazio/dmMPO}{\underline{https://github.com/Ethazio/dmMPO}}\\
  L2 informatique, Groupe C\\
  Faculté des Sciences\\
  Université de Montpellier}
\date{\today}             


\begin{document}

\maketitle


\section{Introduction}

% Rappeler le sujet, ce qu'il fallait faire, l'organisation et le rythme de travail de notre groupe

  Le but de ce devoir est de modéliser puis implémenter en Java le suivi de la gamme de sandwichs d'une boutique de vente de plats à emporter, notamment leur pain, leur sauce et les autres ingrédients les composant.

  Pour cela, nous avons utilisé le logiciel <<StarUML>> pour modéliser le diagramme de classes, <<Eclipse>> pour la programmation en Java, et nous avons créé une zone de dépôt git sur la plateforme GitHub pour synchroniser notre travail en salle de TP et chez nous.

  Au niveau du rythme de travail, nous étions présents lors des TP du vendredi matin pour avancer sur le travail et poser des questions aux professeurs pour éclaircir le sujet. Nous travaillions parallèlement individuellement chez nous, mais aussi ensemble en dehors des cours, à la BU ou l'un chez l'autre.

\section{Installation du projet}

% Étapes nécessaires pour le déploiement de notre programme en local

  Afin d'installer le projet et ainsi pouvoir le tester, vous aurez besoin de tous les fichiers \emph{.java} téléchargés sur Moodle ou sur notre dépôt GitHub et d'un IDE (par exemple Eclipse).

  Procédure d'installation pour Eclipse :
  \begin{enumerate}
    \item
    Ouvrez Eclipse dans n'importe quelle \emph{workspace}
    \item
    Cliquez sur <<File>> $\rightarrow$ <<Open Projects from File System...>>
    \item
    Dans <<Import source>>, cliquez sur <<Directory...>> et sélectionnez le répertoire du projet téléchargé
    \item
    Enfin, cliquez sur <<Finish>>
  \end{enumerate}
  
  Vous pouvez ainsi démarrer le programme par défaut \emph{Main.java} qui illustre les fonctions demandées dans le sujet du devoir en simulant une commande chez une chaîne de magasin de sandwichs connue.


\section{Choix de modélisation}

\emph{TODO}
% Expliquer les différents choix que l'on a pris (ex. quantifier les IAliment avec Ingrédient...)


\section{Conclusion}

\emph{TODO}
% Est-ce qu'on a réussi, ce qui n'a pas marché, ce que le projet nous a appris

\end{document}
